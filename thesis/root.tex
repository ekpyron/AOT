\documentclass[a4paper,enabledeprecatedfontcommands]{scrreprt}
\usepackage{isabelle,isabellesym}
\usepackage{url}
\usepackage{xr}
\usepackage[usenames]{color}
\usepackage{csquotes}
\usepackage{graphicx}
\usepackage{geometry}
\usepackage{epigraph}

\newcommand{\embeddedstyle}[1]{{\color{blue}#1}}

\externaldocument[PM-]{external}

% further packages required for unusual symbols (see also
% isabellesym.sty), use only when needed

\usepackage{amssymb}
  %for \<leadsto>, \<box>, \<diamond>, \<sqsupset>, \<mho>, \<Join>,
  %\<lhd>, \<lesssim>, \<greatersim>, \<lessapprox>, \<greaterapprox>,
  %\<triangleq>, \<yen>, \<lozenge>
\usepackage{amsthm}

\usepackage{amsmath}

%\usepackage{eurosym}
  %for \<euro>

%\usepackage[only,bigsqcap]{stmaryrd}
  %for \<Sqinter>

%\usepackage{eufrak}
  %for \<AA> ... \<ZZ>, \<aa> ... \<zz> (also included in amssymb)

%\usepackage{textcomp}
  %for \<onequarter>, \<onehalf>, \<threequarters>, \<degree>, \<cent>,
  %\<currency>

% this should be the last package used
\usepackage{pdfsetup}

% urls in roman style, theory text in math-similar italics
\urlstyle{rm}
\isabellestyle{it}

% for uniform font size
\renewcommand{\isastyleminor}{\isastyle}

% theorem environments
\newtheorem{remark}{Remark}
\numberwithin{remark}{chapter}
\newtheorem{TODO}{TODO}
\numberwithin{TODO}{chapter}

\title{An Embedding of the Theory of Abstract Objects in Isabelle/HOL}
\author{Daniel Kirchner}

\begin{document}

\begin{titlepage}
\vspace{1cm}

\begin{center}
    \includegraphics[width=0.6\textwidth]{logo}
    \vspace{1cm}


Master's thesis at the institute of mathematics at Freie Universit\"at Berlin

    \vspace{2cm}


    \Large{\textsf{An Embedding of the Theory of Abstract Objects in Isabelle/HOL}}

    \vspace{2cm}

    \large{\textbf{Daniel Kirchner}}

    \vspace{2cm}

    \large{\textbf{
        Supervisor:\\
PD Dr. Christoph Benzm\"uller
    }}

    \vspace{2cm}
    \large{Berlin, \today}
\end{center}
\end{titlepage}
\begin{abstract}
	We present an embedding of the second order fragment of the Theory of Abstract Objects as described in Edward Zalta's
  upcoming work Principia Logico-Metaphysica (PLM\cite{PM}) in the automated reasoning framework Isabelle/HOL. The Theory of Abstract
  Objects is a metaphysical theory that reifies property patterns, as they for example occur in the abstract reasoning
  of mathematics, as \emph{abstract objects} and provides an axiomatic framework that allows to reason about these objects.
  It thereby serves as a fundamental metaphysical theory that can be used to axiomatize and describe a wide range of philosophical
  objects, such as Platonic forms or Leibniz's concepts, and has the ambition to function as a foundational theory of mathematics.
  The target theory of our embedding as described in chapters 7-9 of PLM\cite{PM} employs a modal relational type theory as
  logical foundation for which a representation in functional type theory is known to be challenging\cite{rtt}.
  
  Nevertheless we arrive at a functioning representation of the theory in the functional logic of Isabelle/HOL based on a semantical
  representation of an Aczel model of the theory. Based on this representation we construct an implementation of the deductive
  system of PLM (\cite[Chap. 9]{PM}) which allows it to automatically and interactively find and verify theorems of PLM.

  Our work thereby supports the concept of shallow semantical embeddings of
  logical systems in HOL as a universal tool for logical reasoning as
  promoted by Christoph Benzm\"uller (TODO: reference).

  With the aid of our embedding we were able to uncover a paradox in the formulation of the Theory of Abstract objects
  and to immediately offer several options to modify the theory, such that they becomes provable consistent. Thereby
  our work could provide a significant contribution to the development of a proper grounding of object theory.

\end{abstract}
\tableofcontents

% sane default for proof documents
\parindent 0pt\parskip 0.5ex

% generated text of all theories
\input{Thesis}

\newgeometry{margin=1in}

\appendix
\chapter{Isabelle Theory}
\input{TAO_1_Embedding}
\input{TAO_2_BasicDefinitions}
\input{TAO_3_Semantics}
\input{TAO_4_MetaSolver}
\input{TAO_5_Quantifiable}
\input{TAO_6_Identifiable}
\input{TAO_7_Axioms}
\input{TAO_8_Definitions}
\input{TAO_9_PLM}
\input{TAO_10_PossibleWorlds}
\input{TAO_98_ArtificialTheorems}
\input{TAO_99_SanityTests}

\restoregeometry

% optional bibliography
\bibliographystyle{abbrv}
\bibliography{root}

\end{document}

%%% Local Variables:
%%% mode: latex
%%% TeX-master: t
%%% End:
